\documentclass[12pt,]{report}
\usepackage{lmodern}
\usepackage{amssymb,amsmath}
\usepackage{ifxetex,ifluatex}
\usepackage{fixltx2e} % provides \textsubscript
\ifnum 0\ifxetex 1\fi\ifluatex 1\fi=0 % if pdftex
  \usepackage[T1]{fontenc}
  \usepackage[utf8]{inputenc}
\else % if luatex or xelatex
  \ifxetex
    \usepackage{mathspec}
  \else
    \usepackage{fontspec}
  \fi
  \defaultfontfeatures{Ligatures=TeX,Scale=MatchLowercase}
\fi
% use upquote if available, for straight quotes in verbatim environments
\IfFileExists{upquote.sty}{\usepackage{upquote}}{}
% use microtype if available
\IfFileExists{microtype.sty}{%
\usepackage[]{microtype}
\UseMicrotypeSet[protrusion]{basicmath} % disable protrusion for tt fonts
}{}
\PassOptionsToPackage{hyphens}{url} % url is loaded by hyperref
\usepackage[unicode=true]{hyperref}
\PassOptionsToPackage{usenames,dvipsnames}{color} % color is loaded by hyperref
\hypersetup{
            pdftitle={Book Example},
            pdfauthor={Associação Brasileira de Jurimetria},
            colorlinks=true,
            linkcolor=Maroon,
            citecolor=Blue,
            urlcolor=Blue,
            breaklinks=true}
\urlstyle{same}  % don't use monospace font for urls
\usepackage{natbib}
\bibliographystyle{apalike}
\usepackage{longtable,booktabs}
% Fix footnotes in tables (requires footnote package)
\IfFileExists{footnote.sty}{\usepackage{footnote}\makesavenoteenv{long table}}{}
\usepackage{graphicx,grffile}
\makeatletter
\def\maxwidth{\ifdim\Gin@nat@width>\linewidth\linewidth\else\Gin@nat@width\fi}
\def\maxheight{\ifdim\Gin@nat@height>\textheight\textheight\else\Gin@nat@height\fi}
\makeatother
% Scale images if necessary, so that they will not overflow the page
% margins by default, and it is still possible to overwrite the defaults
% using explicit options in \includegraphics[width, height, ...]{}
\setkeys{Gin}{width=\maxwidth,height=\maxheight,keepaspectratio}
\IfFileExists{parskip.sty}{%
\usepackage{parskip}
}{% else
\setlength{\parindent}{0pt}
\setlength{\parskip}{6pt plus 2pt minus 1pt}
}
\setlength{\emergencystretch}{3em}  % prevent overfull lines
\providecommand{\tightlist}{%
  \setlength{\itemsep}{0pt}\setlength{\parskip}{0pt}}
\setcounter{secnumdepth}{5}
% Redefines (sub)paragraphs to behave more like sections
\ifx\paragraph\undefined\else
\let\oldparagraph\paragraph
\renewcommand{\paragraph}[1]{\oldparagraph{#1}\mbox{}}
\fi
\ifx\subparagraph\undefined\else
\let\oldsubparagraph\subparagraph
\renewcommand{\subparagraph}[1]{\oldsubparagraph{#1}\mbox{}}
\fi

% set default figure placement to htbp
\makeatletter
\def\fps@figure{htbp}
\makeatother

\usepackage[brazilian]{babel}
\usepackage[utf8]{inputenc}
\usepackage[T1]{fontenc}
\usepackage{lipsum}
\usepackage{fullwidth}
\usepackage{indentfirst}
\usepackage[left=2.5cm, right=2.5cm, top=4cm, bottom=3.8cm]{geometry}
\renewcommand{\familydefault}{\sfdefault}
\PassOptionsToPackage{dvipsnames}{xcolor}
\RequirePackage{xcolor} % [dvipsnames]
\definecolor{halfgray}{gray}{0.55} % chapter numbers will be semi transparent .5 .55 .6 .0
\definecolor{webgreen}{rgb}{0,.5,0}
\definecolor{webbrown}{rgb}{.6,0,0}
%\definecolor{Maroon}{cmyk}{0, 0.87, 0.68, 0.32}
%\definecolor{RoyalBlue}{cmyk}{1, 0.50, 0, 0}
%\definecolor{Black}{cmyk}{0, 0, 0, 0}
\usepackage{fancyhdr}
\usepackage{pdfpages}
\usepackage{amsmath}
\usepackage{graphicx}
\usepackage{listings}
\usepackage{enumitem}
\usepackage{setspace}
\usepackage{spverbatim}
\usepackage{lipsum}
\usepackage{natbib}
\usepackage{longtable}
\usepackage{booktabs}
\usepackage{background}

\newcommand{\prestadorEmpresaFoot}{Associação Brasileira de Jurimetria}
\newcommand{\prestadorEmpresa}{Associação Brasileira de Jurimetria}
\newcommand{\prestadorRepr}{Marcelo Guedes Nunes}
\newcommand{\prestadorEnderecoFoot}{Rua Gomes de Carvalho, 1356, 2º andar. CEP 04547-005 - São Paulo, SP, Brasil.}
\newcommand{\prestadorEndereco}{Rua Gomes de Carvalho, 1356, 2º andar}
\newcommand{\prestadorEnderecoComp}{CEP 04547-005 - São Paulo, SP, Brasil}
\newcommand{\prestadorSite}{\url{http://abj.org.br}}
\newcommand{\prestadorEmail}{contato@abj.org.br}
% \includegraphics[trim=left bottom right top, clip]{file}
\newcommand{\logo}{\includegraphics[width=0.21\textwidth, trim=0cm 0cm 0cm 11.1cm, clip]{imgs/logo_abj.png}}

\newcommand{\tipoTrabalho}{}
\newcommand{\numero}{XXX}

\newcommand{\clienteEmpresa}{XXX}
\newcommand{\clienteEndereco}{XXX}
\newcommand{\clienteEnderecoComp}{XXX}
\newcommand{\clienteRepr}{XXX}
\newcommand{\clienteEmail}{XXX}

% \fancypagestyle{firststyle}
% {
%     \pagestyle{fancy}
%     \lhead{\thepage}
%     \chead{}
%     \rhead{\logo{}}
%     \cfoot{
%         \footnotesize{\prestadorEmpresaFoot{}} \\
%         \footnotesize{\prestadorEnderecoFoot{}} \\
%         \footnotesize{\prestadorSite{}}
%     }
%     \renewcommand{\headrulewidth}{0.5pt}
%     \renewcommand{\footrulewidth}{0.5pt}
%     \setlength{\headsep}{.5in}
% }
%
% \pagestyle{firststyle}

\setlength{\parindent}{2em}
% \setlength{\parskip}{1em}
% \renewcommand{\baselinestretch}{1.2}
% % \usepackage{palatino}
% \renewcommand{\familydefault}{\sfdefault} % sans serif
% \fontfamily{ppl}\selectfont

%\renewcommand{\baselinestretch}{1.4}

\backgroundsetup{
scale=1,
angle=0,
opacity=1,
color=black,
contents={\begin{tikzpicture}[remember picture,overlay]
\node at ([xshift=-4.35cm,yshift=-2.5cm] current page.north east) % Adjust the position of the logo.
{\logo}; % logo goes here
\end{tikzpicture}}
}

\usepackage{float}
\let\origfigure\figure
\let\endorigfigure\endfigure
\renewenvironment{figure}[1][2] {
    \expandafter\origfigure\expandafter[H]
} {
    \endorigfigure
}

\title{Book Example}
\author{Associação Brasileira de Jurimetria}
\date{25 de janeiro de 2018}

\begin{document}
\maketitle

{
\hypersetup{linkcolor=black}
\setcounter{tocdepth}{2}
\tableofcontents
}
\listoftables
\listoffigures
\chapter{Introduction}\label{introducao-2}

Para que o seu \texttt{bookdown} funcione tanto na web quanto no pdf,
você deve evitar usar marcadores que dependem do contexto.

Para fazer citações você deve usar \citep{Weinstein1997} ou
\citet{Weinstein1997}. Isso também funciona pra pacotes \citep{R-base}
ou \citet{R-base}. Para criar uma figura, é preferível que você use o
\texttt{print} padrão do \texttt{knitr}. A label do gráfico será
\texttt{fig:label-do-chunk}. Você pode citar fazendo
\ref{fig:label-do-chunk}.

\begin{figure}
\centering
\includegraphics{bookdown_files/figure-latex/label-do-chunk-1.pdf}
\caption{\label{fig:label-do-chunk}Este é um gráfico.}
\end{figure}

Se você precisar importar uma imagem de fora do R, é melhor que você
faça \texttt{!{[}{]}()}, a despeito do que diz o Yihui.

\begin{figure}
\centering
\includegraphics{imgs/logo_abj.png}
\caption{Essa é a caption.}\label{fig:logo}
\end{figure}

Se você estiver com muita vontade de seguir os ensinamentos do mestre,
você pode usa o \texttt{knitr::include\_graphics}, mas vai precisar
setar \texttt{dpi\ =\ NA}.

\begin{figure}

{\centering \includegraphics{imgs/logo_abj} 

}

\caption{The RStudio addin to help input LaTeX math.}\label{fig:mathquill}
\end{figure}

Essa forma tem três vantagens:

\begin{enumerate}
\def\labelenumi{\arabic{enumi}.}
\tightlist
\item
  A label fica no mesmo formato das demais.
\item
  Você pode setar o \texttt{fig.height} e o \texttt{fig.width}.
\item
  A numeração vai funcionar corretamente no \texttt{html}.
\end{enumerate}

O problema é que esse jeito não é o natural do \texttt{pandoc}.

De toda forma, não é uma boa criar imagens que vieram dentro de pastas.
Você terá problemas com o \texttt{path} dos arquivos. Ao invés de
simplesmente copiar as pastas, você precisa fazer isso manualmente.

Outro tipo de referência interessante é a referência a subseções. Você
pode usar \protect\hyperlink{objetivos}{essa sintaxe} pra ir pra seção
de objetivos. Você também pode usar \ref{objetivos}.

Por fim, pra inserir tabelas, não use nada. O \texttt{abjTemplates} faz
tudo por você. As captions, inclusive, agora podem ser adicionadas no
cabeçalho do \texttt{chunk}. O \texttt{abjTemplates} sabe o que fazer
dependendo do \texttt{output}.

\chapter{Introdução}\label{introducao}

\hypertarget{objetivos}{\section{Objetivos}\label{objetivos}}

\section{Hipóteses}\label{hipoteses}

\section{Resultados Esperados}\label{resultados-esperados}

\section{Organização do trabalho}\label{organizacao-do-trabalho}

\chapter{Metodologia}\label{metodologia}

\section{Bases de dados}\label{bases-de-dados}

\section{Modelos}\label{modelos}

\chapter{Resultados}\label{resultados}

\bibliography{bibliography/book.bib,bibliography/packages.bib}

\end{document}
